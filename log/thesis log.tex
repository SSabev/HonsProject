\documentclass[11pt]{amsart}
\usepackage{geometry}                % See geometry.pdf to learn the layout options. There are lots.
\geometry{letterpaper}                   % ... or a4paper or a5paper or ... 
%\geometry{landscape}                % Activate for for rotated page geometry
%\usepackage[parfill]{parskip}    % Activate to begin paragraphs with an empty line rather than an indent
\usepackage{graphicx}
\usepackage{amssymb}
\usepackage{epstopdf}
\DeclareGraphicsRule{.tif}{png}{.png}{`convert #1 `dirname #1`/`basename #1 .tif`.png}

\title{Thesis log}
\author{Stefan Sabev}
%\date{}                                           % Activate to display a given date or no date

\begin{document}
\maketitle

\section{14th October 2014}

What's been tried so far:

\begin{itemize}
\item I have used smoothing to smooth the weekly seasonality component out of both the twitter and searches data.
That has yielded small improvement in the correlation coefficients.
\item I have also used a very basic method of prediction which works as follows:
\begin{quotation}
Calculate the mean and the standard deviation of the searches.
If the standard deviation is more than the mean, then there has been a spike which has pushed it higher.
\end{quotation}
\item I've also used that to determine which destinations should have a classifier built. That has yielded very small improvements.
\end{itemize}


The fact that the simple combination of LASSO + Ridge regression does not perform miraculously well has led my supervisor and I to believe that perhaps we should investigate more sophisticated models that will perhaps model the problem better.

I am currently doing:

\begin{itemize}
\item Reading the Adams \& MacKay paper on Bayesian change point models
\item Will look into the matlab implementation and try to port it to Python.
\end{itemize}


\end{document}  