\documentclass[bsc,frontabs,twoside,singlespacing,parskip]{infthesis}

\usepackage[xetex]{graphicx}
\usepackage{fontspec,xunicode}
\defaultfontfeatures{Mapping=tex-text,Scale=MatchLowercase}
\setmainfont[Scale=.95]{Georgia}
\setmonofont{Georgia}

\begin{document}

\title{Modelling search volumes as a dynamic system responding to external events}

\author{Stefan Sabev}

\course{Master of Informatics}
\project{{\bf MInf Project (Part 1) Report}}

\date{\today}

\abstract{
It is well known that some events might spark people's interest to fly to different destinations. In particular news events or sports events can quite easily make people search for a specific destination - for example the Champions League Quarter final draw increased the number of flight searches from Glasgow to Spain 6 times.
The main goal of this project is to grab Twitter data and possibly the BBC news web-site and extract the available event data. Afterwards we'd want to split it into two groups - those who spark people's interest and those don't.
This can later be used to detect when people want to go somewhere based on the news channels around the world.
}

\maketitle

\section*{Acknowledgements}
Acknowledgements go here. 

\tableofcontents

\pagenumbering{arabic}


\chapter{Introduction}


The document structure should include:
\begin{itemize}
\item
The title page  in the format used above.
\item
An optional acknowledgements page.
\item
The table of contents.
\item
The report text divided into chapters as appropriate.
\item
The bibliography.
\end{itemize}

Commands for generating the title page appear in the skeleton file and
are self explanatory.
The file also includes commands to choose your report type (project
report, thesis or dissertation) and degree.
These will be placed in the appropriate place in the title page. 

The default behaviour of the documentclass is to produce documents typeset in
12 point.  Regardless of the formatting system you use, 
it is recommended that you submit your thesis printed (or copied) 
double sided.

The report should be printed single-spaced.
It should be 30 to 60 pages long, and preferably no shorter than 20 pages.
Appendices are in addition to this and you should place detail
here which may be too much or not strictly necessary when reading the relevant section.

\section{Using Sections}

Divide your chapters into sub-parts as appropriate.

\section{Citations}

Note that citations 
(like \cite{P1} or \cite{P2})
can be generated using {\tt BibTeX} or by using the
{\tt thebibliography} environment. This makes sure that the
table of contents includes an entry for the bibliography.
Of course you may use any other method as well.

\section{Options}

There are various documentclass options, see the documentation.  Here we are
using an option ({\tt bsc} or {\tt minf}) to choose the degree type, plus:
\begin{itemize}
\item {\tt frontabs} (recommended) to put the abstract on the front page;
\item {\tt twoside} (recommended) to format for two-sided printing, with
  each chapter starting on a right-hand page;
\item {\tt singlespacing} (required) for single-spaced formating; and
\item {\tt parskip} (a matter of taste) which alters the paragraph formatting so that
paragraphs are separated by a vertical space, and there is no
indentation at the start of each paragraph.
\end{itemize}

\chapter{Data gathering}

\section{Citations}

\chapter{Analysis}

\section{Hashtags}

Generic ones - NO

Place names - much better


% use the following and \cite{} as above if you use BibTeX
% otherwise generate bibtem entries
\bibliographystyle{plain}
\bibliography{mybibfile}

\end{document}
