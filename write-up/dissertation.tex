\documentclass[bsc,frontabs,twoside,singlespacing,parskip]{infthesis}

\usepackage[xetex]{graphicx}
\usepackage{fontspec,xunicode}
\defaultfontfeatures{Mapping=tex-text,Scale=MatchLowercase}
\setmainfont[Scale=.95]{Georgia}
\setmonofont{Georgia}

\begin{document}

\title{Modelling search volumes as a dynamic system responding to external events}

\author{Stefan Sabev}

\course{Master of Informatics}
\project{{\bf MInf Project (Part 1) Report}}

\date{\today}

\abstract{
It is well known that some events might spark people's interest to fly to different destinations. In particular news events or sports events can quite easily make people search for a specific destination - for example the Champions League Quarter final draw increased the number of flight searches from Glasgow to Spain 6 times.
The main goal of this project is to grab Twitter data and possibly the BBC news web-site and extract the available event data. Afterwards we'd want to split it into two groups - those who spark people's interest and those don't.
This can later be used to detect when people want to go somewhere based on the news channels around the world.
}

\maketitle

\section*{Acknowledgements}
Acknowledgements go here. 

\tableofcontents

\pagenumbering{arabic}


\chapter{Introduction}

In recent times social media has been of great interest to everyone. Everyone is trying to benefit from the vastness of the data available - companies are paying for sentiment analysis, targeted ads, promoted trends and so on. 

With a base of 190 million active monthly users, Twitter{\copyright}  has turned into quite an promising source of data. Many are trying harness the power of social media and use it to some purpose - to see how their product or company is performing or predict something about the future such as flu outbreaks, etc.

Here in this dissertation we are looking at the effects of social media on online travel. The effects haven't been explored in such depth before. We aim to show that using Twitter as the main source we are able to predict certain shifts in demand on some particular routes, which are reflected in social media. 

\chapter{Data gathering}

\section{Citations}

\chapter{Analysis and exploration of the data set}

\section{Hashtags}

Generic ones - NO

\section{Mentions} 

Place names - much better

\chapter{Finding the correlation between those variables}


% use the following and \cite{} as above if you use BibTeX
% otherwise generate bibtem entries
\bibliographystyle{plain}
\bibliography{mybibfile}
%\begin{thebibliography}{50}
%\bibitem{TwitStats} http://www.statisticbrain.com/twitter-statistics/ \textsl{Twitter stats}
%\end{thebibliography)

\end{document}
